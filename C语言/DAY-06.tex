 problem 15:


//组合问题


int64_t ans = 1, m = 20;
for(int64_t i = 40; i >= 21; i--)
{
	ans *= i;
	while(m && ans % m == 0)
	{
		ans /= m;
		m--;
	}
}


任何一种满足结合律的运算都可以使用到快速幂;


0 1 2
3 4 5    ->
6 7 8


A[I][K] * A[K][J]   从I点出发经过K步可以走到J点  



void output(int matr[][9])

{
	char clo_style[] = "%4d";
	char num_style[] = "%4d";
	printf("\n    |");
	int ret = 5;
	for(int i = 0; i < 9; i++)
	{
		ret += printf("%4d", i);
	}
	printf("\n");
	for(int i = 0; i < 9; i++)
	{
		printf(col_style, i);
		for(itn j = 0; j < 9; j++)
		{
			printf(num_style, matr[i][j]);
		}
		printf("|\n");
	}
	return;
	
}


int main()

{

	int matr[9][9] = {0};
	for(int i = 0; i < 9; i++)
	{
		for(int j = i + 1; j < 9; j++)
		{
			if(j - i == 1 && j % 3 == 0)
			continue;
			if(j - i != 1 && j - i != 3)
			continue;
			mati[i][j] = 1;
			matr[j][i] = 1; 
		}
	
	}
}



problem 18:

动态规划基础

// 运筹学

动态规划 是 递推的一个子集

动态规划是求最优解

//   步骤    动guifangcheng   状态转移方程
1.   F(I, J);  ->   max(F[I - 1][J] + NUM, F[I - 1][J - 1] + NUM)  阶段划分
(1) 从顶 到 I,J点获取到的最大值
(2) 从  I, J 到底端的最大值


f[0][0] = num[0][0];
for(int i = 1; i < max; i++)
{
	for(int j = 0; j <= i; j++)
	f[i][j] = f[i - 1][j] + num[i][j];
	if(j == 0|| f[i - 1][j - 1] <= f[i][j]) continue;
	f[i][j] = f[i - 1][j - 1] + num[i][j];
}

int ans = 0;
for(int i = 1; i < max; i++)
{
	if(f[max - 1][i] > ans)
	ans = f[max - 1][i];

}




problem 24:

康拓展开



1.线性晒
2.大整数乘法
3.因子和










