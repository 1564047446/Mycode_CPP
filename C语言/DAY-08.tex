for(初始化; 循环条件; 执行后操作)
{
	代码块; //  符合语句	
}


int main() // 特殊的主函数


递归程序的组成部分
1、边界条件处理
2、针对问题的处理过程和递归过程
3、结果返回

头递归和尾递归为啥返回值不一样;

/*指针函数*/内存不仅可以存放数据还可以存放  /*代码段*/ -> 也是一种数据 
int (*)(int) //函数指针类型
int (*a)(int)  //函数指针类型的变量

/*变参函数*/
#include<stdarg.h>
int max_int(int a, ...)

va_list 获得a往后的参数列表
va_start 定位a后面第一个参数的位置
va_arg 获取下一个可变参数列表中的参数
va_end结束整个获取可变参数列表的动作

/*实现*/  //test44.cpp//

int max_int(int num, ...)
{
	va_list arg;
	va_start(arg, num);
	int ans = 0;
	while(num--)
	{
		int t = va_arg(arg, int);
		if(t > ans)
			ans = t;
	}
	va_end(arg);
	return ans;
}


#define haizei_arg(pp, T)({ \
	char * temp1 = (char *)(pp); \
	T *temp2 = (T *)(pp); \
	temp1 += sizeof(T); \
	pp = temp1; \
	*temp; \  /*返回值*/
})

int main()
{
	char buffer[100];
	for(int i = 0; i < 100; i++)
	{
		buffer[i] = i % 16;
	}
	double *p1 = (double *)(buffer + 4);
	*p1 = 123.456;
	double *p = (double *)buffer;
	int a = haizei_arg(p, int);
	printf("%d %d %p\n", a, a, p);
	double b = haizei_arg(p, double);
	printf("%lf %p\n", b, p);
	return 0;
}


