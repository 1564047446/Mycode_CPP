

problem 45:

// 函数指针    int qqq(int x, int(*func)(int))



problem 09:

1. 他们的整数倍也是勾股数,(na, nb, nc)也是勾股数
2. (a, b, c)之间两两互质
3.  a, b 必为一个奇数一个偶数
4. 任何素勾股数均可表示为如下星星, 其中n < m, 切/*gcd(n, m) = 1;*/
    a = 2 * n * m;
    b = m ^ 2 - n ^ 2;
    c = m ^ 2 + n ^ 2;



// 素勾股数



problem 12:

//任意非素数的整数等于 n个素数的x幂的乘积的形式   比如: 18 = 2 ^ 1 * 3 ^ 2;

// A与B为互质的数 C = A * B;   因子个数应该等于 (Pa) * (pb);


for(int i = ; i * i <= x; i++)
{

	if(x % i == 0)
	continue;
	int times = 0;
	while(x % i == 0)
	{
		x /= i;
		times++;
		
	}
	num *= times;
}
	if(x != 1)
	num *= 2;
	
	//  利用素数筛框架 达成时间复杂度为O(1)
	
int min_num[];  //最小素因子的幂数
int num[];    //因数的个数
int prime[];   //素数表


for(int i = 2; i <= max_range; i++)
{
	if(!prime[i])
	{
		prime[ ++prime[0]] = i;
		min_num[i] = 1;
		num[i] = 2;
	}
	for(int j = 1; j <= prime[0]; j++)
	{
		if(i * p[j] > max_range)
		break;
		if(i % p[j] != 0)
		{
			num[prime[j] * i] = num[prime[j]] * num[i];
			min_num[prime[j] * i] = 1;
		}
		else
		
		{
			num[prime[j] * i] = num[i] / (min_num[i] + 1) * (min_num[i] + 2);
			min_num[prime[j] * i] = min_num[i] + 1;
			break;
			
		}
	}
}


problem 21:




