
<set> 容器使用
头文件
#include <set>
标准库
std::set;
定义一个set容器
set<int> sets;
正常定义的容器自定义是从小到大
如果想重大到小的进行插入元素
set<int, greater<int> > coll;
set容器不允许有重复元素
	插入元素
	sets.insert(1);
	
<mutiset> 无序容器
头文件
#include <unordered_set>
标准库
std::unordered_mutiset;
定义一个mutiset
	unordered_mutiset<int> coll;
	coll.insert({1, 2, 3 ,4, 5, 5, 3}); C++11 允许这么插入
	
	
<vector> <deque> <list> <forward_list>  扩容操作
	vector<int> coll;
	coll.resize(size);

移除元素
	list<int> coll;
	list<int>::iterator end = remove(coll.begin(), coll.end(), 3);
	end 返回的移除动作后的新的逻辑终点
	distance 获取删除元素的数量
	distance(end, coll.end());
	把需要删除的元素斩草除根 erase
	coll.erase(end, coll.end())
	
对于高效的开发,尽量使用成员函数(STL)

	for_each()函数针对区间每一个元素,调用一个由用户制定的函数
	void print(int p) {
		cout << p << endl;
	}
	for_each(coll.begin(), coll.end(), print);
	
函数对象比正常函数效率要高
	通过重载操作符() 去完成函数对象
	这么写的好处是,一个函数可以处理多种状态
	class AddValue {
	private: 
		int value;
	public:
		AddValue(int v) : value(v) {}
		
		void operator() (int& elem) const {
			elem += this->value;
		}
	
	}
	`每个元素加上10`
	for_each(coll.begin(), coll.end(), AddValue(10))
	
	
transform 算法
给定运算规则,将两个容器内的元素处理后的结果写入第三个容器中	

lambda表达式 是一种隐式的预定义函数对象和函数对象一样快


	
